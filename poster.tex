\documentclass{tikzposter}

% \settitle{\vbox{
%      \color{titlefgcolor} {\bfseries \Huge \sc \@title \par}
%      \vspace*{1em}
%      {\LARGE \@author \par} \vspace*{1em} {\large \@institute}
% }}

\makeatletter
\renewcommand\TP@maketitle{%
   \begin{minipage}{0.8\linewidth}
        \color{titlefgcolor}
        {\bfseries \Huge \sc \@title \par}
        \vspace*{1em}
        {\huge \@author \par}
        \vspace*{1em}
        {\LARGE \@institute}
    \end{minipage}
    \hfill
    \begin{minipage}{0.2\linewidth}
       \centering
       \includegraphics[width=16cm]{uoa}
    \end{minipage}
}
\makeatother

\title{Efficient Bayesian Evolutionary Analysis using Hamiltonian Monte Carlo}
\author{\textbf{Arman Bilge}, Timothy Vaughan, and Alexei J. Drummond}
\institute{email: \texttt{armanbilge@gmail.com}}

\usetheme{Autumn}
\usecolorstyle{Russia}
\renewcommand{\familydefault}{\sfdefault}
\usepackage{sfmath}

% Surface plot
\tikzset{>=latex}
\usetikzlibrary{arrows.meta}
\usepackage{pgfplots}
\pgfplotsset{width=30cm,compat=newest}

\usepackage{mathtools}
\newcommand{\dd}{\, \mathrm{d}}
\renewcommand{\vec}[1]{\ensuremath{\boldsymbol{\mathbf{#1}}}}
\newcommand{\mat}[1]{\ensuremath{\boldsymbol{\mathbf{#1}}}}
\newcommand{\op}[1]{\ensuremath{\boldsymbol{\mathbf{#1}}}}
\newcommand{\norm}[1]{\ensuremath{\mathcal{N}\left(#1\right)}}

\usepackage{algorithm}
\usepackage{algpseudocode}

\begin{document}

    \maketitle

    \begin{columns}

        \column{0.5}

        \block{Motivation}{
            \begin{itemize}
                \item Bayesian evolutionary analysis is a powerful tool for rate estimation, divergence dating, phylogeography, species tree estimation, etc.
                \item Usually implemented with the Markov chain Monte Carlo algorithm (MCMC)
                \item MCMC has poor performance
                \item Hamiltonian Monte Carlo (HMC)
            \end{itemize}
        }

        \block{Hamiltonian Dynamics}{
            Let $\pi\left(\vec{q}\right) \equiv P\left(\vec{\theta}\mid\vec{D}\right)$ be the target probability density.

            Consider a particle with position $\vec{q}$ and momentum $\vec{p}$.

            Then the Hamiltonian for our system is
            \begin{equation}
                H\left(\vec{q},\vec{p}\right) = U\left(\vec{q}\right) + K\left(\vec{p}\right)
            \end{equation}
            with potential energy
            \begin{equation}
                U\left(\vec{q}\right) = -\log{\pi\left(\vec{q}\right)}
            \end{equation}
            and kinetic energy
            \begin{equation}
                K\left(\vec{p}\right) = \frac{1}{2} \vec{p}^T \mat{M} \vec{p}
            \end{equation}
            where $\mat{M}$ represents the particle's mass.

            The system's dynamics are described by Hamilton's equations
            \begin{align}
                \frac{\dd \vec{q}}{\dd t} &= \frac{\partial H}{\partial \vec{p}}, \\
                \frac{\dd \vec{p}}{\dd t} &= -\frac{\partial H}{\partial \vec{q}},
            \end{align}
            which are integrated to find the state at a particular time.
        }

        \block{The HMC Algorithm}{

            The three operators.

            \innerblock{Flip Operator}{
                Flips the particle's momentum.
                \begin{equation}
                    \op{F}\left\{\vec{q},\vec{p}\right\} = \left\{\vec{q},-\vec{p}\right\}
                \end{equation}
                \centering
                \begin{tikzpicture}[scale=1.5]
                    \draw [very thick] (-3,0) -- (3,0);
                    \draw [very thick] (0,-3) -- (0,3);
                    \draw [arrows={*->[scale=100]}, line width=1mm] (-2,-1) -- (-1,-0.125);

                    \draw[-{>[scale=0.25]}, line width=1cm] (4,0) -- (5.5,0);
                    \node at (4.6,1) {$\op F$};
                \end{tikzpicture}
                \qquad
                \begin{tikzpicture}
                    \begin{axis}[width=10cm,
                                 axis x line=center,
                                 axis y line=center,
                                 ticks=none,
                                 xmin=-3,
                                 xmax=3,
                                 ymin=-3,
                                 ymax=3]
                        \addplot [line width=1mm, *->] plot coordinates {(-2,-1) (-1,-0.125)};
                    \end{axis}
                \end{tikzpicture}
            }

            \innerblock{Leapfrog Operator}{
                Simulates the motion of the particle for some time $s=\epsilon L$ by making $L$ leapfrog steps of size $\epsilon$.
                \begin{equation}
                    \op{L}\left\{\vec{q}^t,\vec{p}^t\right\} = \left\{\vec{q}^{t+s},\vec{p}^{t+s}\right\}
                \end{equation}
            }

            \innerblock{Momentum Randomization Operator}{
                Randomizes the particle's momentum.
                \begin{equation}
                    \op{R}\left\{\vec{q},\vec{p}\right\} = \left\{\vec{q}, \sqrt{1-\alpha}\vec{p} + \sqrt{\alpha}\vec{n}\right\},
                    \vec{n} \sim \norm{\vec{0}, \mat{M}^{-1}},
                \end{equation}
            }

            \begin{algorithmic}[1]
            \Function {HamiltonUpdate}{$\left\{\vec{q},\vec{p}\right\}$}
                \State $\left\{\vec{q}^\prime, \vec{p}^\prime\right\}
                    \leftarrow \op{F}\op{L}\left\{\vec{q},\vec{p}\right\}$
                \State $a \leftarrow \min\left(1,
                    \exp\left(
                        H\left(\vec{q}, \vec{p}\right) - H\left(\vec{q}^\prime,
                            \vec{p}^\prime\right)\right)\right)$
                \State $\left\{\vec{q},\vec{p}\right\} \leftarrow
                    \begin{cases}
                        \left\{\vec{q}^\prime, \vec{p}^\prime\right\}
                            & \text{with probability } a \\
                        \left\{\vec{q},\vec{p}\right\}
                            & \text{with probability } 1 - a
                    \end{cases}$
                \State $\left\{\vec{q},\vec{p}\right\} \leftarrow
                            \op{R}\op{F}\left\{\vec{q},\vec{p}\right\}$
                \State \Return $\left\{\vec{q},\vec{p}\right\}$
            \EndFunction
            \end{algorithmic}
        }

        \column{0.5}

        \block{HMC versus MCMC}{

            \centering
            % \begin{tikzfigure}
                \begin{tikzpicture}
                    % Tree
                    \draw [line width=1mm] (2.5,16.5) -- (2.5,18.5);
                    \draw [line width=1mm] (5.5,16.5) -- (5.5,18.5);
                    \draw [line width=1mm] (2.5,18.5) -- (5.5,18.5);
                    \draw [line width=1mm] (4,18.5) -- (4,20.5);
                    \draw [line width=1mm] (8.5,16.5) -- (8.5,20.5);
                    \draw [line width=1mm] (4,20.5) -- (8.5,20.5);
                    \draw [line width=1mm] (6.25,20.5) -- (6.25,21.5);
                    \draw [ultra thick, dashed] (2.5,16.5) -- (10,16.5);
                    \draw [ultra thick, dashed] (2.5,18.5) -- (10,18.5);
                    \draw [ultra thick, dashed] (2.5,20.5) -- (10,20.5);
                    \node at (10,17.5) {$t_2$};
                    \node at (10,19.5) {$t_1$};

                    % Legend
                    \draw [line width=1mm, red] (16,22) -- (17,22);
                    \draw [line width=1mm, violet] (16,21) -- (17,21);
                    \node [text=red, right] at (17.125,22) {\textbf{HMC}};
                    \node [text=violet, right] at (17.125,21) {\textbf{MCMC}};

                    \begin{axis}[colormap/greenyellow,
                                 mesh/interior colormap={graygray}{color=(gray) color=(gray)},
                                 /pgf/number format/1000 sep={},
                                 view={72}{32},
                                 xlabel=$t_1$,
                                 ylabel=$t_2$,
                                 zlabel=$U\left(\vec{q}\right)$,
                                 zlabel style={rotate=-90},
                                 xtick={0,0.2,...,2},
                                 ytick={0,0.2,...,1},
                                 ztick={1000,1020,...,1200}
                                 ]
                        \addplot3[surf] table {surface.dat};
                        \addplot3[violet, mark=*, line width=1mm] table {mcmc.dat};
                        \addplot3[red, mark=square*, line width=1mm] table {hmc.dat};
                    \end{axis}
                \end{tikzpicture}
            % \end{tikzfigure}
        }

        \block{Future Directions}{
            \begin{itemize}
                \item Automatic tuning of parameters
            \end{itemize}
        }

        \block{Acknowledgements}{
            SMBE Undergraduate Travel Award \\
            \includegraphics[width=10cm]{awc}
            \includegraphics[width=10cm]{nesi}
        }

        \block{References}{
            \hangindent=50pt Alexei J. Drummond et al. `Bayesian phylogenetics with BEAUti and the BEAST 1.7'. In: \emph{Mol Biol Evol} 29.8 (2012).

            \hangindent=50pt Simon Duane et al. `Hybrid Monte Carlo'. In: \emph{Physics Letters B} 195.2 (1987).

            % Joseph Felsenstein. `Evolutionary Trees from DNA Sequences: A Maximum Likelihood Approach'. In: \emph{J Mol Evol} 17.6 (1981).

            \hangindent=50pt Nicholas Metropolis et al. `Equation of State Calculations by Fast Computing Machines'. In: \emph{J Chem Phys} 21.6 (1953).

            \hangindent=50pt Radford M. Neal. `MCMC Using Hamiltonian Dynamics'. In: \emph{Handbook of Markov Chain Monte Carlo}. Boca Raton, Florida: Chapman and Hall/CRC, 2011.

            \hangindent=50pt Eric E. Schadt et al. `Computational Advances in Maximum Likelihood Methods for Molecular Phylogeny'. In: \emph{Genome Research} 8.3 (1998).
        }

    \end{columns}

\end{document}